
\documentclass[11pt]{article} % use larger type; default would be 10pt

\usepackage[utf8]{inputenc} % set input encoding (not needed with XeLaTeX)

%%% PAGE DIMENSIONS
\usepackage{geometry} % to change the page dimensions
\geometry{a4paper} % or letterpaper (US) or a5paper or....

\usepackage{graphicx} % support the \includegraphics command and options


\usepackage{verbatim} % adds environment for commenting out blocks of text & for better verbatim
\usepackage{subfig} % make it possible to include more than one captioned figure/table in a single float
% These packages are all incorporated in the memoir class to one degree or another...

%%% HEADERS & FOOTERS
\usepackage{fancyhdr} % This should be set AFTER setting up the page geometry
\pagestyle{fancy} % options: empty , plain , fancy
\renewcommand{\headrulewidth}{0pt} % customise the layout...
\lhead{}\chead{}\rhead{}
\lfoot{}\cfoot{\thepage}\rfoot{}

%%% SECTION TITLE APPEARANCE
\usepackage{sectsty}
\allsectionsfont{\sffamily\mdseries\upshape} % (See the fntguide.pdf for font help)
% (This matches ConTeXt defaults)

%%% ToC (table of contents) APPEARANCE
\usepackage[nottoc,notlof,notlot]{tocbibind} % Put the bibliography in the ToC
\usepackage[titles,subfigure]{tocloft} % Alter the style of the Table of Contents
\renewcommand{\cftsecfont}{\rmfamily\mdseries\upshape}
\renewcommand{\cftsecpagefont}{\rmfamily\mdseries\upshape} % No bold!

%%% END Article customizations

%%% The "real" document content comes below...

\title{Validation of the CMS Hadron Calorimeter Phase I Data Acquisition System}
\author{Andrew Whitbeck, Ken Call, Jim Hirschauer}

\begin{document}
\maketitle

\abstract{An overview of upgrades to the CMS Hadron Calorimeter (HCal) data acquisition system will be presented. 
Design changes to both the frontend and backend electronics will be descussed.  Performance of pro  duction backend
electronics and prototype frontend electronics will be presented.  Results from integration tests from both bench top
and beam test experiments will be presented.  }

\section{Introduction}

CMS is a great undertaking and the Hadronic Calorimeter (HCal) plays a crucial role in the physics program
of CMS.  blah blah blah ...

This note is organized as follows.  Section~\ref{overview} will discuss the current CMS hadronic calorimeter 
and the design components that will be upgraded as well as the motivations for these upgrades.   
Sections~\ref{frontend} and~\ref{backend} will describe the upgraded readout electronics, comparing and
contrasting them to the existing electronics.  Section~\ref{results} will present results from bench top and beam 
integration tests which verify the functionality of different key desing components.  Finally, Section~\ref{conclusion}
will summarize the successes of the project and work yet to be done. 

\section{Overview of Upgrades}

The CMS Hadronic Calorimeter consists of four major subdetectors, the barrel region ($|\eta|<x.xx$) , 
the endcap region ($x.xx|\eta|<3.0$), the forward region ($3.0<|\eta|<5.0$), and the outer region.   
During the last portion of LHC run I, these subdetectors used three separate technologies.  

The barrel and endcap regions consisted of brass absorber and organic plastic scintillators as the active 
material.  The photodetectors collecting sampled light were hybrid photodiodes (HPD).  
Except in regions were towers overlapped, 
all longitudinal segments were optically summed before being read out.  

In the forward calorimeter (HF), the abosber material is steel and the active material is a set of long and 
short quartz fibers running parrellel to the beam line.  The long and short segment allow for both hadronic
and electromagnetic components of showers to be distinguished.  The quartz fibers are optically coupled 
to photomultiplier tubes (PMT).  

Existing readout electronics consist of frontend electronic which reside either on the detector (HB, HE, HO)
or in electronics racks just next to the detector (HF).  The design principles for the readout is the same for
all subdetectors although due to physical constraints, the implementation is slightly different.  The frontend
electronics consist of a custom Application Specific Integrated Circuit (ASIC) which digitizes analog signals
from the photodetectors.  This ASIC is known as the Charge (Q) Integrator and Encoder (QIE) is an ADC which
can integrate at 40 Mhz which a dynamic range that spans XXX orders of magnitude of integrated charge 
while minimizing the quantization error \ref{QIE}.   Digitized signals from the QIE are then formatted by a
second ASIC and routed to the backend electronics. 

%%% describe the CCM

In the backend racks, data is recieved by the HCal Trigger and Readout board.  This is a VME style board buffers data and communicates with the global calorimeter trigger.  Trigger primitives are computed using a set of lookup tables to convert charge into energy.  Once a bunch crossing is accepted by the level-1 trigger system, the data is finally sent to the Data Concentrator Card which builds events and forwards these to the central DAQ system for full reconstruction.

\subsection{Design Motivations for the Phase I Upgrade}

In the barrel endcap region where HPDs are utilized, the effect of the strong magnetic field has been found
to have a dramatic effect on the long term performance of the HPDs.   Figure~\ref{} shows the dispersion of 
HPD gain as a function of time as a result of prolonged exposure to magnetic fields.  The HPDs have has
been found to discharge causing large anomalous signal to be readout.  These effect were more prominently 
notice in the HO region where the magnetic field is large due to its proximity with the steel return yolk.  As a 
result, Silicon Photomultipliers (SiPM) have been install and operated in HO since, ???.  Replacing the HPDs
in the barrel and endcap subdetectors will not only remedy the performance loss, but will also allow for
better resistance to radiation exposure due to the higher gain of the SiPMs and increased longitudinal
due to the reduced size of SiPMs.  However, this will dramatically increase the number of channels which are 
readout.  

In the forward region muons or shower punch through are know to produce high rates of anomalous signals 
which arrive out of time from shower signals.  These signals arise from high energy particles moving through
the borosilicate glass covering the photocathode of the PMT and producing Cerenkov light.  Due to the bunch 
structure of the LHC run I, 50 ns spacing, CMS was able to tune the charge integration window in order to move 
almost all of these anomalous signals into the previous bunch crossing.  Otherwise, these signal would present 
a major challenge for both triggering and reconstruction.  Aside from being out of time, these signal are also 
distinct from signals from particle showers in how the photons are distributed on the surface of the photocathode.
Thus, making use of the mutlianode PMTs, anomalous signals can be rejected based on how assymetric the 
signals on different pairs of anodes are.  This, however, requires that the adapter boards within the PMT boxes 
which electrically sum the four anodes will need to be adapted such that two pairs of anodes are electrically 
summed.  This further requires that the number of channels readout in the HF subdetector double.  

\subsection{Installation Schedule}

Upgrades of the PMT in HF have already been successfully completed during the ongoing long shutdown.
Cabling for the dual anode readout has been laid out.  Backend electronics are currenly being produced and tested
at CERN with installation to be completed by the end of LS1.  Adapter boards and frontend electronics
will be installed during the year end technical stop (YETS) between the years 2015 and 2016.  Prototype frontend
electronics have been produced and preproduction boards are scheduled to be produced by the end of the
year 2014.  This schedule ensures that the backend electronics will already have been commissioned before 
the frontend electronics are installed relieving the work load during the start of collisions in the year 2016.  

The barrel and endcap frontend electronics require that the detector be opened in order to install.  As a results, 
these components are not scheduled to be installed until the extended YETS of the year 2017.    As such, prototype 
frontend electronics are not scheduled to be built before 2015.  However, since the design is largely the same 
as for the HF subdetector, most of the design validation can be completed long before then using the HF components.  



\section{Frontend Electronics}

In each of the individual subdetectors, the frontend electronics are grouped into sets of n QIE cards which 
house different numbers of QIEs each.  For the barrel and endcap, each group resides in a readout box (RBX),
which is an aluminum housing containing cooling mechanics and a custom backplane for power and signal 
distribution.  In HF, each group will reside in a eurostyle crate with a custom backplane.  In each of the RBXs and
HF crates, there will also exist an ngCCM and a calibration card.  Table~\ref{table:frontendComponents} shows how many QIE 
cards and channels will be readout per RBX and HF crate. 


\begin{table}
\begin{center}
\begin{tabular}{|l|c|c|c|}
\hline
Component & HF & HB & HE \\
\hline \hline
QIE version & 10 & 11 & 11 \\ \hline
QIEs & 24 & 12 & 12 \\ \hline
QIE cards per RBX/crate & 12 & 16 & 16 \\ \hline 
\end{tabular}
\caption{Tabulation of different components which will be used in each of the different HCal subsystems.}
\label{table:frontendComponents}
\end{center}
\end{table}

\subsection{Charge Integrator and Encoder ASIC}

The new QIE ASIC will include larger dynamic range, a TDC, and programmable monitoring capabilities.  
Separate versions will be used for HF and HB/HE subsystems, see Table~\ref{table:frontendComponents} for
details.  The TDC has a 6-bit encoding which is readout over 3 double data rate (DDR) pins.  The ADC has an 8-bit
encoding and is read out over 4 DDRs.  Each QIE has the same set of range selector and integrator duplicated 
four times to provide deadtimeless integration.  Each of these subcircuits has a Cap-ID accosiated with it which is 
readout with a single DDR.  Certain functionality of the QIE is programmable, such as the input pedestal DAC, 
through a programmable 64 bit shift register.  

\subsection{QIE Card}

The QIE card will contain the upgraded QIE ASICs.  Each QIE card will digitize and format data from 12 (24)
channels for the HB/HE (HF) subsystem.  The data is captured by a commercial Microsemi Igloo2 FPGA.  
Once captured, the TDC discriminant can be used to compute falling edge TDC.  The ADC and rising edge
TDC information for 6 separate channels is formatted into a 12 byte data frame and the 12 falling edge TDC 
measurements are formed into a separate 12 byte frame.  These frames are sent to 3 separate serializers 
which are located on the Igloo2.  Serialized data is then sent, via a Versatile Tranceiver (VTTx)~\ref{VTTXref}, 
to a $\mu$HTR at 5.8 Gbps.  Each VTTx module can drive 2 fibers.  In total, there are 2 (3) VTTx modules 
on each QIE card.  

Power distribution on the QIE card is handle using a set of custom DC/DC converters... something, is this 
really necessary.

\subsection{Next-Generation Clock, Control, and Monitoring Card}

The ngCCM card will manage the clock signal, slow and fast control signals, and monitoring functionality.  One 
ngCCM will manage an entire RBX (crate) in the HB/HE (HF) subsystems.  This single card will forward the clock
to each QIE card within the RBX (crate) as well as $I^2C$ signal and reset signals.  As such, all control and 
monitoring is facilitated by the ngCCM.  

\section{Backend Electronics}

The backend electronics for the HCal systems will reside in the service cavern.  All backend electronics will
reside in commercial Vadatek $\mu$TCA crates.  These crates will be power by... A crate will consiste of 12 
$\mu$HTRs, an AMC13, and a $\mu$TCA Control Hub (MCH).  Except for the latter, these devices were custom
designed for data acquisition in particle physics.

\subsection{$\mu$HTR}

The $\mu$HTR two main purposes, receiving and buffering data before forwarding data to the AMC13 and communicating trigger primitives to the global calorimeter trigger system.  

Data is received from the frontend via a fiber bundle into an Avago AFBR-820BEPZ PPOD optical receiver.  
% show figure 6.3 from HCal upgrade TDR
These optical receivers are capable of accepting 12 fibers, a single QIE card pack or a pair of HF QIE cards (???)
and are capable of operating up to 10 Gbps

Electrical signals from the PPOD recievers are deserialized using a Xilinx Virtex 6-series FPGA.  The data is then
synchronized and trigger primitives are computed for the level-1 trigger system.  The data is then pipelined 
until a level-1 accept signal is received.  

An SDRAM buffer is also included to allow of latency in the back FPGA.  This buffer is 64 MB which is sufficient
for a 100 ms latency at the fully DAQ rate, x.xx hz.  

IPBus communication with both the front and back FPGA is enabled via the GbE line.  This allows for register 
control.  

Figure~\ref{} shows the block diagram for the  $\mu$HTR.  

\subsection{AMC13}

\section{Setup}

\subsection{LED tests}

\subsection{Internal charge injection tests}

\section{Results}

A suite of test have been carried out to validate the design and performance of the HF frontend electronics.  
These tests have been carried out using both a bench top setup and particle beam tests using a mock HF 
calorimeter.

\subsection{Slow control communication}

The QIE ASICS and IGLOO2 FPGA are intended to be configured and monitored via I$^2$C buses.  The
bridge FPGA acts as a mux for this communication and routes the signals to the appropriate location.  
Users can select between either of the 2 IGLOO2 FPGAs or between any of the 4 chains of 6 QIEs.  details 
on the list of registers that can be written to and read from as well as the details on the communication
protocols can be found in Ref.~\cite{BridgeSpecs}.  

To test this functionality and robustness of the I$^2$C communication with both the IGLOO2 registers and the
QIE shift registers, both systematic and random values were written to register and read back over an extended 
period of time.  ... Show results of this test.

\subsection{QIE shift registers}

In order to verify that the shift registers for the QIEs are being properly written to and read from by the QIEs, 
a number of test which result in predictable QIE behavior were carried out.  

Fixed range mode test.  QIE pedestal DAC.  Anymore?

\subsection{TDC validation}

\subsection{ADC validation}

\subsection{QIE cross talk}

\subsection{QIE clock phase}

\subsection{JTAG programming over backplane}

\subsection{Double pulses}

\subsection{$\mu$HTR optical link BERT}

\end{document}
